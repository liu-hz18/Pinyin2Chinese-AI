
\documentclass[UTF8, onecolumn, a4paper]{article}
\usepackage{ctex}
\setlength{\parindent}{2em}
\usepackage{appendix}
\usepackage{geometry}
\usepackage{amsmath, amsthm}
\usepackage{multirow, multicol}
\usepackage{subfigure}
\usepackage{float}
\usepackage{graphicx}
\usepackage{lettrine}
\usepackage{authblk}
\usepackage{indentfirst}
\usepackage{xcolor, fontspec}%用于设置颜色
\usepackage[ruled,vlined]{algorithm2e}
\usepackage{listings}%用于显示代码
\usepackage[colorlinks,
linkcolor=black,
anchorcolor=blue,
citecolor=green
]{hyperref}
\usepackage{tikz}
\usetikzlibrary{trees}
\geometry{left=2.5cm,right=2.5cm,top=1.7cm,bottom=1.8cm}
\newfontfamily\code{Consolas} %代码用字体

\title{\textbf{拼音输入法: 实验报告}}%———总标题
\author{刘泓尊\quad 2018011446\quad 计84}
\date{}
%\affil{Department of Computer Science, Tsinghua University}

\begin{document}
\maketitle
\tableofcontents
\lstset{%代码块全局设置
	backgroundcolor=\color{red!3!green!3!blue!3},%代码块背景色为浅灰色
	rulesepcolor= \color{gray}, %代码块边框颜色
	breaklines=true,  %代码过长则换行
	numbers=left, %行号在左侧显示
	numberstyle= \small,%行号字体
	%keywordstyle= \color{red},%关键字颜色
	commentstyle=\color{gray}, %注释颜色
	frame=shadowbox,%用方框框住代码块
	xleftmargin=1em,
	xrightmargin=0em,
	tabsize=5,
	%rulesepcolor=\color{red!20!green!20!blue!20},  %阴影颜色
	keywordstyle={\color{blue!90!}\fontspec{Consolas Bold}},   %关键字颜色
	commentstyle={\color{blue!70!black}\fontspec{Consolas Italic}},   %注释颜色
	stringstyle=\color{orange!100!black}, %字符串颜色
	numberstyle=\color{purple}, %行号颜色
	%basicstyle=\ttfamily, %代码风格
	basicstyle=\fontspec{Consolas},
	showstringspaces=false,          % underline spaces within strings only  
	showtabs=false,
	captionpos=t, %文件标题位置
	flexiblecolumns
}
\clearpage
本项目完整文件与模型已经上传到\href{https://cloud.tsinghua.edu.cn/d/e9d7a28fd1174a1689da/}{清华云盘 https://cloud.tsinghua.edu.cn/d/e9d7a28fd1174a1689da/},总大小约5GB(其中语料集2.1GB)。如果您想跳过约30min的模型构建时间,请从云盘上下载训练好的模型。运行/bin下的pinyin.py即可开始测试。具体操作与功能见本报告。请不要更改文件路径,以免错误;\textbf{如果您想重新构建训练模型,请先运行“clean.py”}将json文件转化为纯文本,之后使用“\textbf{python pinyin.py -p}”实现预训练,训练结束后,您可以使用“\textbf{python pinyin.py inputfile outputfile [answerfile]}”进行测试。如:
\begin{center}
“python pinyin.py ../input/input.txt ../output/output.txt ../answer/answer.txt”  
\end{center}
                                                                           
\section{原理简介}
\paragraph*{}
我分别实现了2-Gram和2-Gram,3-Gram结合的方式进行,基于Viterbi算法进行时间复杂度的优化。
\subsection{问题建模}
基于统计知识,由一个拼音串转化为汉字串的模型可以抽象为求解使得
$$P(S) = P(w_0w_1\cdots w_n) = \prod_{i=1}^{n}P(w_i\mid w_0w_1\cdots w_{i-1})$$
最大的候选句子序列$w_0w_1\cdots w_n$。根据大数定律,条件概率$P(w_i\mid w_0w_1\cdots w_{i-1})$可以使用频率近似代替,即
$$P(w_i\mid w_0w_1\cdots w_{i-1}) = \frac{Count(w_0w_1\cdots w_{i-1}w_i)}{Count(w_0w_1\cdots w_{i-1})}$$
上述模型是隐马尔可夫模型(HMM)。
\subsection{2-gram模型}
2-gram模型假设句子中某个单词的出现取决于其前面一个单词,对应于概率模型中的
$$P(S) = \prod_{i=1}^{n}P(w_i\mid w_{i-1}) = \prod_{i=1}^{n}\frac{Count(w_{i-1}w_i)}{Count(w_{i-1})}$$
例如对于句子“<s> I am Sam </s>”,上述模型为\\
P("<s> I am Sam </s>") = P(i $\mid$ <s>)P(am $\mid$i)P(Sam $\mid$am)P(</s> $\mid$Sam)
\subsection{Backoff and Interpolation}
考虑到可能存在$Count(w_{i-1}w_i) = 0$的情况,会产生概率为0的情形。将上式修正为:
$$P(w_i\mid w_{i-1}) = \alpha \cdot\frac{Count(w_i)}{\sum_{i}Count(w_i)} + (1-\alpha)\cdot \frac{Count(w_{i-1}w_i)}{Count(w_{i-1})}$$
有论文称之为“Backoff and Interpolation”(退化与插值)方法。经过多次权衡,我取$\alpha = 0.05$.
\subsection{3-gram模型}
3-gram模型假设句子中下一个单词的数先取决于前面两个单词。即
$$P(S) = \prod_{i=2}^{n}P(w_i\mid w_{i-2}w_{i-1}) = \prod_{i=2}^{n}\frac{Count(w_{i-2}w_{i-1}w_i)}{Count(w_{i-2}w_{i-1})}$$
为了防止状态转移过程中出现概率为零,将上式修正为:
$$P(w_i\mid w_{i-2}w_{i-1}) = \alpha \cdot\frac{Count(w_i)}{\sum_{i}Count(w_i)} + \beta \cdot \frac{Count(w_{i-1}w_i)}{Count(w_{i-1})} + \gamma\cdot \frac{Count(w_{i-2}w_{i-1}w_i)}{Count(w_{i-2}w_{i-1})}$$
其中$\alpha + \beta + \gamma = 1$.经过多次测试,最终取$\alpha = 0.05, \beta = 0.85, \gamma = 0.10$.
\subsection{Viterbi算法}
Viterbi算法是求解HMM的有效优化算法,该算法基于动态规划思想。记dp[t][i]为第t个字取候选汉字序列第i个字的概率。那么有$$dp[t][i] = \arg\max_{j\in words(t-1)} \left(dp[t-1][j]\times P(w_i\mid w_j)\right)$$
根据此思想,设每个拼音的候选汉字最大为$T$,计算第t层$T$个节点概率的复杂度为$O(T^2)$, 那么求解整个句子$(length(S) = n)$的时间复杂度为$O(nT^2)$
\subsection{Viterbi算法扩展到3-gram}
3-gram模型的每个状态和前面两层有关,所以可以增加一个维度信息,记dp[t][j][i]为第t个位置的汉字在上一个字为j的条件下,取i的概率。那么Viterbi算法可以修正为:
$$dp[t][j][i] = \arg\max_{k\in words(t-2)}\left(dp[t-1][k][j]\times P(w_i\mid w_kw_j)\right)$$
由于综合考虑了前面两步的信息,所以该算法的复杂度提升到了$O(nT^3)$.但是因为$T$是固定的,对于每个拼音,其对应的汉字一般只有10-20个,所以该算法可以看做是$O(n)$的.
\paragraph*{}
为了节省模型空间,我只保留了出现频率前$10^6$的三元组$(w_k, w_j, w_i)$.而不是一个三维矩阵。为了可以使用矩阵存储,我将字按字频进行排序并编号,字频统计结果放在./sina\_news\_vocab/vocab.txt中。
\subsection{一些优化}
\subsubsection{加入句子起始符<bos>与结束符<eos>}
为了保证句子的完整性,准确预测具体开头与结尾的汉字,我引入了起始符<bos>和结束符<eos>.并在统计过程中,以标点符号作为句子起始结束的标志,进行2-gram和3-gram的统计。在跑Viterbi算法时,计入<bos>和<eos>的影响。经过测试,该改进有效提升了句子第一个字和最后一个字的准确率。如将“禁用的武侠小说非常精彩”改进为“金庸的武侠小说非常精彩”。
\subsubsection{Beam Search}
朴素的Viterbi算法状态数非常多,3-gram模型高达$O(nT^2)$.为了改进时间复杂度,加快搜索速度,我使用Beam Search进行优化。Beam Search Decoding的思路是: 在第t层不保存所有的候选节点,而是只保存概率最大的前s个节点,剩余的节点不被传播到下一层。如果只保留概率前$T/2$个节点,那么2-gram的时间复杂度可以减小为$O(nT^2/4)$.
\subsubsection{Unknown Words}
对于不在“一二级单词表”中的汉字,我将其标记为<unk>,以提高算法的普适性。该改进对算法的准确率提升并无作用,但是却保留了句子的完整性。
\subsection{其他尝试}
\subsubsection{Laplace平滑}
我还尝试了其他的平滑化方法,比如Laplace平滑,该平滑使用
$$P_{Laplace}(w_i\mid w_{i-1}) = \frac{Count(w_{i-1}w_i) + 1}{Count(w_{i-1})+V}$$
其中$V$为词汇表的大小。但是测试发现效果不如Backoff and Interpolation平滑,我最终没有采用此方法。
\subsubsection{语料库的扩展}
为了观察语料库对模型的影响,我爬取了一部分微博语料作为训练集。经过测试发现,加入微博语料之后,口语测试集的准确率有了明显上升(约5\%的改进),但是书面语测试集的准确率有所下降。我查阅了相关文献,发现在基于统计的框架下,这两个领域暂时无法“优雅地”兼顾。因此,作为一种取舍,最终我使用新闻数据集,保证了书面语的准确率最高。
\subsubsection{Interpolation(插值法)参数的确定}
我使用了可以有效确定$\alpha, \beta, \gamma$的比例的方法,称为Deleted Interpolation Algorithm(删除插值法).该方法可以通过统计词频来调整三个参数的比例。具体流程如下:\\
\begin{algorithm}[H]
	\caption{Deleted Interpolation Algorithm}%算法名字
	%\LinesNumbered %要求显示行号
	\KwIn{$corpus$;}%输入参数
	\KwOut{$\alpha$, $\beta$, $\gamma$;}%输出
	$\alpha = 0, \beta = 0, \gamma = 0$;\\ %\;用于换行
	\For{each 3-gram $w_1, w_2, w_3$ with  $Count(w_1w_2w_3) > 0$} {
		\textbf{depending} on the maximum of the following 3 values:\\
		\textbf{case} $\frac{Count(w_1w_2w_3)-1}{Count(w_1w_2)-1}$: increment $\gamma$ by $Count(w_1w_2w_3)$\\
		\textbf{case} $\frac{Count(w_2w_3)-1}{Count(w_2)-1}$:      increment $\beta$ by $Count(w_1w_2w_3)$\\
		\textbf{case} $\frac{Count(w_3)-1}{N - 1}$:                increment $\alpha$ by $Count(w_1w_2w_3)$\\
	}
	Normalize $\alpha$, $\beta$, $\gamma$\\
	\Return $\alpha$, $\beta$, $\gamma$
\end{algorithm}

\section{具体功能与使用流程}
运行过程包括(1).数据预处理, (2).模型训练和(3).测试 三个阶段。在预处理和训练过程中会新建文件作为存档,在测试阶段会加载这些文件,用于计算输出结果。具体流程为:
\subsection{语料预处理}
\paragraph*{}
由于给出的数据时\code{json}格式,并且带有很多冗余信息,所以我先将语料文件转换为“纯文本”,以便后续训练过程。该过程将字典中\code{"html"}项的内容抽取出来,并替换其中的“标点符号”为句子的开始符\code{E}和结束符\code{B}。例如:
“清华大学计算机系。”$\rightarrow$“E清华大学计算机系B”
\paragraph*{}
您可以在\code{./sina\_news\_gbk/}下执行\code{python clean.py}进行上述过程。新得到的“纯文本”文件命名为\code{2016-xx\_con.txt},仍然放\code{./sina\_news\_gbk/}下。为了识别标点符号,我使用了\code{zhon.hanzi}库下的\code{punctuation}变量。
\subsection{模型训练}
\paragraph*{}
我分别实现了2-gram模型和2-gram与3-gram结合的方式。所以预处理过程主要是统计2-gram和3-gram的频率。在本目录下运行python pinyin.py -p可以开始模型构建过程。注意,对应的语料2016-xx\_con.txt需要放在./sina\_news\_gbk/下。对于课程提供的语料库,构建过程约为30min,最坏情况下占用5GB内存资源.
\subsection{模型测试}
\paragraph*{}
模型测试需要使用python pinyin.py inputfile outputfile [answerfile]格式,其中inputfile是拼音文件,每行一句拼音。转换结果将输出到outputfile。如果您需要验证正确率,请将标准答案放在answerfile下,并在命令中加入此参数,转换完成后将自动开始正确性的检验,分别输出字正确率和句正确率。
\subsection{从stdin输入测例}
\paragraph*{}
您也可以从stdin输入合法的拼音句子,得到对应的输出。请使用python pinyin.py -i开启该功能。注意,您同样需要在./sina\_news\_gbk/下准备好2016-xx\_con.txt语料。
\subsection{文言文模型}
\paragraph*{}
我本来想利用给定的语料实现对文言文和诗歌的翻译功能,但是查阅文献后得知,古文和现代汉语是两个不同的语言体系,其语言特征差距较大,很多时候不能达到“鱼和熊掌兼得”的效果。因此,我单独实现了文言文拼音翻译模块。我爬取了\href{https://www.gushiwen.org/}{古诗文网}下的几万首古诗文,经过同样的预处理过程得到了古文语料poem.txt.之后基于此语料进行训练,模型保存在./vocab\_poem文件夹下。您可以使用此模块进行对文言文的测试,效果很好。
\subsection{查看帮助}
\paragraph*{}
为了帮助使用者尽快掌握使用流程,您可以使用python pinyin.py -h查看help信息。具体介绍为:
\begin{lstlisting}[language={}, title={help}] 
$ python pinyin.py -h

If you HAVE NOT preprocessed data, please run:
	python pinyin.py -p

Attention:
This Preprocess will takes for about 30 mins and 5GB Runtime memory!!!

If you HAVE preprocessed data, please run:
	python pinyin.py inputfile outputfile [answerfile]

A simple demo is:
	python pinyin.py ./input.txt ./output.txt ./answer.txt

If you want to run from stdin, please input
	python pinyin.py -i

If you want to run from stdin which translating [poems and classical chinese], please input
	python pinyin.py -c
\end{lstlisting}
\paragraph{文件结构}
\tikzstyle{every node}=[draw=black,thick,anchor=west]
\tikzstyle{selected}=[draw=red,fill=red!30]
\tikzstyle{optional}=[dashed,fill=gray!50]
\begin{center}
	\begin{tikzpicture}
	[
	grow via three points={one child at (0.5,-0.7) and
		two children at (0.5,-0.7) and (0.5,-1.4)},
	edge from parent path={(\tikzparentnode.south)  |-(\tikzchildnode.west)}]
	\node {2018011446}
	child { node {bin} }
	child { node {input} }
	child { node {output} }
	child { node {answer} }
	child { node {sina\_news\_gbk} }
	child { node {sina\_news\_vocab} }
	child { node {vocab\_poem} }
	child { node {拼音汉字表\_12710172} }
	child { node {刘泓尊\_2018011446\_实验报告} };
	\end{tikzpicture}
\end{center}


\section{准确率统计}
\paragraph{二元模型: }正确例子\\
$\bullet$请大家选择你觉得可以的时间\\
$\bullet$美军方称不承认中国东海防空识别区\\
$\bullet$特朗普希望不久和中国国家主席面对面会晤
\paragraph{二三元结合:}新增正确例子\\
$\bullet$金庸的武侠小说非常精彩\\
$\bullet$智能技术与系统国家重点实验室\\
$\bullet$对染色体人工合成的工作给予了高度评价\\
$\bullet$你的世界会变得更精彩
\paragraph{二三元结合:}不好的例子\\
$\bullet$他养了\textbf{一致青瓦}当宠物\\
$\bullet$\textbf{他}是我的母亲\\
$\bullet$今天回家比较\textbf{完}\\
$\bullet$中国\textbf{诗人}人民民主专政的社会主义国家
\paragraph{古文模型:} 好的例子\\
$\bullet$举头望明月低头思故乡\\
$\bullet$君不见黄河之水天上来奔流到海不复回
\paragraph{错例分析}
第1个错误和第3个准确体现了“语料特征”,因为语料是新闻语料,“青瓦”的概率远大于“青蛙”的概率,而“比较完”的概率也明显多于口语化的“比较晚”; 第2个错误在基于统计的模型下难以解决,因为“他”的出现概率高于“她”,实际上,搜狗输入法面对多个“ta”的问题,也是给出了若干候选,不能根本上保证概率最高的结果就是正确的;第3个错误则体现了2-gram对模型的过度影响,模型对汉字的选择明显偏向于“二元词”而非“单字词”。
\paragraph{准确率} 我统计了不同模型和参数的准确率,效果如下:
\begin{center}
\begin{tabular}{|c|c|c|c|}
	\hline
	模型 & 二元模型 & 三元模型 & 二三元结合 \\
	\hline
	口语集: 字准确率 & 0.9259 & 0.6011 & \textbf{0.9264} \\
	\hline
	口语集: 句准确率 & 0.5661 & 0.4221 & 0.5666 \\
	\hline
	书面语集: 字准确率 & 0.9608 & 0.7142 & \textbf{0.9748} \\
	\hline 
	书面语集: 句准确率 & 0.7678 & 0.5102 & \textbf{0.8170} \\
	\hline
\end{tabular}
\end{center}
二元模型下参数$\alpha$的选择(书面语集):
\begin{center}
\begin{tabular}{|c|c|c|c|c|c|}
	\hline
	alpha & 0.2 & 0.1 & 0.05 & 0.01 & 0.001 \\
	\hline
	字正确率 & 0.9412 & 0.9543 & \textbf{0.9608} & 0.9601 & 0.9532 \\
	\hline
	句正确率 & 0.7518 & 0.7614 & \textbf{0.7678} & 0.7678 & 0.7564 \\
	\hline
\end{tabular}
\end{center}

\textbf{注: }口语集使用微博语料,书面语集使用历年政府工作报告。对应的answerfile使用pypinyin进行转换。规模约为10W个汉字, 1W个句子。可以看到,二三元结合的准确率最高,仅使用三元模型的准确率最低。书面语测试集准确率明显高于口语测试集。二三元结合的方式甚至可以与搜狗输入法媲美。
\section{改进思路}
尽管二三元结合的模型已经达到了极高的准确率,甚至可以与搜狗输入法媲美,但是我还是思考了若干改进思路。我希望将来有时间实现与验证这些改进。
\subsection{Bidirectionality}
\paragraph*{}
原始的算法仅仅考虑到了下一个单词取决于上文的汉字,却没有考虑到下文的汉字。而汉语体系中,一个汉字是基于上下文(Context-Related)的。所以我认为可以考虑双向的模型,综合“过去”与“未来”的信息。
\paragraph*{}
最简单的思路是,分别统计“从左到右”和“从右到左”两个方向的2-gram和3-gram。使用Viterbi算法分别从两个方向计算句子概率,选择概率最大的那个结果。
\paragraph*{}
经过查阅文献,另一个著名且有效的算法是Conditional Random Field(条件随机场)。
CRF通过构建非有向图模型,将“未来”的句子信息引入当前位置汉字概率的计算。但是该模型的时间复杂度较高,计算过程比HMM缓慢。
\subsection{语料库的适当扩充}
\paragraph*{}
经过实验发现,模型的输出结果和语料特征强相关。由于本模型主要采用新闻语料进行训练,对于口语测试集的准确率明显不如书面语测试集;针对古文的训练也可以看到这个现象。可以采取多个类型的文本,使得模型适用于多种不同场合。
\subsection{尝试更高元模型}
\paragraph*{}
由于汉语中二元词频率占了极大部分,所以本模型使用二三元结合的方式已经可以胜任正常需求。为了进一步提高准确率,可以考虑4-gram模型或者更多元的模型。这样做可能会提高对一些俗语和成语的准确率。
\subsection{基于“词”的n-Gram模型}
可以考虑建立词典,使用分词算法保存出现的多字词,以“词”为单位进行训练和计算。
\subsection{基于Sequence Processing Networks}
从拼音到汉字的转换问题是一个典型的“序列到序列”的问题。因此,可以使用处理序列的RNN, LSTM,GRU等神经网络模型得到语言模型。比如bi-LSTM可以很好的实现序列翻译与Context-Related的功能。
\section{总结}
这是我第一次接触人工智能算法,我收获良多。我实践了NLP领域十分著名的N-Gram模型,了解了马尔科夫过程在人工智能领域的广泛应用,领略了Viterbi算法的明显优势,最终达到了90\%以上的准确率。
\paragraph*{}
同时,我阅读了大量论文,了解了如何使用Beam Search对Viterbi算法进行加速,了解了“条件随机场”在马尔科夫模型中的应用,并使用Deleted Interpolation Algorithm来确定参数比例,学习了Backoff and Interpolation的平滑化方法。同时,我了解了Sequence Processing Networks的前沿知识,这些知识都对我的实现有很大启发,也是对我科研之路的一个启蒙。
\paragraph*{}
写完本项目后,我便开始关注我使用的搜狗输入法。在我写本报告的过程中,搜狗输入法用的算是比较顺手,但是也难免出现预测错误的情况,尤其是对于“专有名词”。但不得不说,拼音输入法确实给人带来了巨大的便利,我也期待着更多的改进!
\paragraph*{}
感谢马老师和助教的悉心指导!
\end{document}